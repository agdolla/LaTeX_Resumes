%-------------------------------------------------------------------------------
%	SECTION TITLE
%-------------------------------------------------------------------------------
\cvsection{Projects}

%-------------------------------------------------------------------------------
%	CONTENT
%-------------------------------------------------------------------------------
\begin{cventries}
	%---------------------------------------------------------
	\cventry
	{{\color{awesome} Unreal Engine}} % Skills used
	{Parallax \begin{footnotesize}{(Link: \href{https://www.youtube.com/playlist?list=PLJCbmN5AjE1XUlgY5KwFnwDh3gmhvtiSC}{\color{awesome}\underline{Development Log}})}\end{footnotesize}} % Project name
	{Personal Project} % Event
	{MAR 2017} % Date
	{
		\begin{cvitems} % Description(s) of experience/contributions/knowledge
			\item {Parallax was my first attempt into true game design and a great learning experience. The end product is the framework for a 3D side-scrolling cover shooter. The game starts in a side-scrolling view that provides greater visibility but lacks accuracy in aiming. Players may switch to an over-the-shoulder 3rd person shooter view in order to accurately aim, at the sacrifice of how far in front of themselves they can see. Along with typical side-scrolling elements, the 3rd dimension of depth into the screen adds another element of concern for the player.}
		\end{cvitems}
	}
	%---------------------------------------------------------
	\cventry
	{{\color{awesome} C\#, Unity}} % Skills used
	{Oculus Drift} % Project name
	{HackYSU} % Event
	{FEB 2017} % Date
	{
		\begin{cvitems} % Description(s) of experience/contributions/knowledge
			\item {OculusDrift was an experiment in audio-visual entrainment employing Unity (C\#) and the Oculus 
				Rift. The purpose of the project was to create a relaxing environment by using binaural audio and
				simulating the user floating through a star field.}
		\end{cvitems}
	}
			
\end{cventries}