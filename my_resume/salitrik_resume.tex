%!TEX TS-program = xelatex
%!TEX encoding = UTF-8 Unicode
% Awesome CV LaTeX Template for CV/Resume
%
% This template has been downloaded from:
% https://github.com/posquit0/Awesome-CV
%
% Author:
% Claud D. Park <posquit0.bj@gmail.com>
% http://www.posquit0.com
%
% Template license:
% CC BY-SA 4.0 (https://creativecommons.org/licenses/by-sa/4.0/)
%
%-------------------------------------------------------------------------------
% CONFIGURATIONS
%-------------------------------------------------------------------------------
% A4 = a4paper, Letter = letterpaper
\documentclass[12pt, a4paper]{awesome-cv}

% Default page margins:
% \geometry{left=1.4cm, top=.8cm, right=1.4cm, bottom=1.8cm, footskip=.5cm}

\geometry{left=1.5cm, top=.75cm, right=1.25cm, bottom=1.5cm, footskip=.5cm}

% Specify the location of the included fonts
\fontdir[fonts/]

% Color for highlights
% Awesome Colors: awesome-emerald, awesome-skyblue, awesome-red, awesome-pink, awesome-orange
%                 awesome-nephritis, awesome-concrete, awesome-darknight

% Custom Awesome Colors: awesome-midnight, awesome-lunar, awesome-sapphire, awesome-blue,
%						 awesome-deepemerald, awesome-pitchblack
\colorlet{awesome}{awesome-midnight}

% Colors for text
% Uncomment if you would like to specify your own color
% \definecolor{darktext}{HTML}{414141}
% \definecolor{text}{HTML}{333333}
% \definecolor{graytext}{HTML}{5D5D5D}
% \definecolor{lighttext}{HTML}{999999}

\definecolor{darktext}{HTML}{111111}
\definecolor{text}{HTML}{111111}
\definecolor{graytext}{HTML}{111111}
\definecolor{lighttext}{HTML}{111111}

% Set false if you don't want to highlight section with awesome color
\setbool{acvSectionColorHighlight}{false}
\setbool{acvEntryHighlight}{false}
\setbool{acvPositionHighlight}{true}


% If you would like to change the social information separator from a pipe (|) to something else
\renewcommand{\acvHeaderSocialSep}{\quad\textbar\quad}


%-------------------------------------------------------------------------------
%	PERSONAL INFORMATION
%	Comment any of the lines below if they are not required
%-------------------------------------------------------------------------------
% Available options: circle|rectangle,edge/noedge,left/right
% \photo[rectangle,edge,right]{./examples/profile}
\name{Kyle Patrick}{Salitrik}
\position{Programmer{\enskip\cdotp\enskip}Electromechanical Engineer}
\address{ADDRESS}

\mobile{+(00)000-000-0000}
\email{xxxxxxxxx@gmail.com}
%\homepage{}
\github{NullFragment}
\linkedin{ksalitrik}
% \gitlab{gitlab-id}
\stackoverflow{789504}{NullFragment}

%-------------------------------------------------------------------------------
\begin{document}

% Print the header with above personal informations
% Give optional argument to change alignment(C: center, L: left, R: right)
\makecvheader[C]

% Print the footer with 3 arguments(<left>, <center>, <right>)
% Leave any of these blank if they are not needed
\makecvfooter
{}%\today}
{Kyle P. Salitrik~~~·~~~Résumé}
{}%\thepage}


%-------------------------------------------------------------------------------
%	CV/RESUME CONTENT
%	Each section is imported separately, open each file in turn to modify content
%-------------------------------------------------------------------------------
%-------------------------------------------------------------------------------
%	SECTION TITLE
%-------------------------------------------------------------------------------
\cvsection{Work Experience}


%-------------------------------------------------------------------------------
%	CONTENT
%-------------------------------------------------------------------------------
\begin{cventries}
	
	%---------------------------------------------------------
	\cventry
	{Undergraduate Research Intern} % Job title
	{The Pennsylvania State University} % Organization
	{University Park, PA, USA} % Location
	{MAY 2017 - PRESENT} % Date(s)
	{
		\begin{cvitems} % Description(s) of tasks/responsibilities
			\item {\textbf{Programming} | {\color{awesome}Python, Keras, Tensorflow, CNTK}
				\begin{itemize}[noitemsep,wide=0pt, leftmargin=\dimexpr\labelwidth + 2\labelsep\relax]
					\item[\textbullet]{Developed Python scripts for automating and randomizing software tests}
					\item[\textbullet]{Trimmed unnecessary libraries from deployed software}
					\item[\textbullet]{Prototyped Logistic and Linear regression Machine Learning Algorithms in Matlab}
					\item[\textbullet]{Used Keras with Tensorflow \& CNTK to prototype Neural Networks}
				\end{itemize}}
		\end{cvitems}
	}
	
	%---------------------------------------------------------
	\cventry
	{Electromechanical Engineer} % Job title
	{Advanced Acoustic Concepts} % Organization
	{Uniontown, PA, USA} % Location
	{MAR 2015 - JAN 2017} % Date(s)
	{
		\begin{cvitems} % Description(s) of tasks/responsibilities
			\item {\textbf{Software \& Hardware Engineering} | {\color{awesome}Test Automation, Arduino, Python, BASH, AutoCAD Electrical}
				\begin{itemize}[noitemsep,wide=0pt, leftmargin=\dimexpr\labelwidth + 2\labelsep\relax]
					\item[\textbullet]{Utilized BASH scripts and Knoppix to automate server tests by printing status information to a built-in LCD panel}
					\item[\textbullet]{Automated mprime CPU stress tests via SSH using Python scripts to deploy the service and collect logs}
					\item[\textbullet]{Designed test fixtures using Arduino microcontrollers to automate hardware testing of CCAs using bitwise control of ICs}
					\item[\textbullet]{Created Arduino array to deliver real-time, self-correcting PWM pulse generation  by polling output}
				\end{itemize}}
			\item {\textbf{Mechanical Engineering} | {\color{awesome}Solidworks, Inventor}
				\begin{itemize}[noitemsep,wide=0pt, leftmargin=\dimexpr\labelwidth + 2\labelsep\relax]
					\item[\textbullet]{Developed adaptable 3-axis vibration test fixture for up to 2U, 30-inch servers and frequency range up to 2kHz}
					\item[\textbullet]{Created vibration test fixture to accommodate various sizes of Hammond enclosures for low-frequency MIL-SPEC testing}
					\item[\textbullet]{Designed modular truss structure for supporting mobile winches on ships with the goal of being hot-swappable for missions}
				\end{itemize}}
		\end{cvitems}
	}
	    
	%---------------------------------------------------------
	\cventry
	{Teaching Intern (Statics)} % Job title
	{The Pennsylvania State University} % Organization
	{University Park, PA, USA} % Location
	{AUG 2012 - DEC 2012} % Date(s)
	{
		\begin{cvitems} % Description(s) of tasks/responsibilities
			\item {Held office hours to help students comprehend subject matter and complete homework}
			\item {Assisted with creation of exam problems and proctored exams}
		\end{cvitems}
	}
	
	
	%---------------------------------------------------------
\end{cventries}
\vspace{-1em}
%-------------------------------------------------------------------------------
%	SECTION TITLE
%-------------------------------------------------------------------------------
\cvsection{Projects}

%-------------------------------------------------------------------------------
%	CONTENT
%-------------------------------------------------------------------------------
\begin{cventries}
	%---------------------------------------------------------
	\cventry
	{{\color{awesome} Unreal Engine}} % Skills used
	{Parallax \begin{footnotesize}{(Link: \href{https://www.youtube.com/playlist?list=PLJCbmN5AjE1XUlgY5KwFnwDh3gmhvtiSC}{\color{awesome}\underline{Development Log}})}\end{footnotesize}} % Project name
	{Personal Project} % Event
	{MAR 2017} % Date
	{
		\begin{cvitems} % Description(s) of experience/contributions/knowledge
			\item {Parallax was my first attempt into true game design and a great learning experience. The end product is the framework for a 3D side-scrolling cover shooter. The game starts in a side-scrolling view that provides greater visibility but lacks accuracy in aiming. Players may switch to an over-the-shoulder 3rd person shooter view in order to accurately aim, at the sacrifice of how far in front of themselves they can see. Along with typical side-scrolling elements, the 3rd dimension of depth into the screen adds another element of concern for the player.}
		\end{cvitems}
	}
	%---------------------------------------------------------
	\cventry
	{{\color{awesome} C\#, Unity}} % Skills used
	{Oculus Drift} % Project name
	{HackYSU} % Event
	{FEB 2017} % Date
	{
		\begin{cvitems} % Description(s) of experience/contributions/knowledge
			\item {OculusDrift was an experiment in audio-visual entrainment employing Unity (C\#) and the Oculus 
				Rift. The purpose of the project was to create a relaxing environment by using binaural audio and
				simulating the user floating through a star field.}
		\end{cvitems}
	}
			
\end{cventries}
\vspace{-1em}
%-------------------------------------------------------------------------------
%	SECTION TITLE
%-------------------------------------------------------------------------------
\cvsection{Education}

%-------------------------------------------------------------------------------
%	CONTENT
%-------------------------------------------------------------------------------
\begin{cventries}
	%---------------------------------------------------------
	\cventry
	{B.S. in Engineering Science \& Computational Data Science} % Degree
	{The Pennsylvania State University} % Institution
	{University Park, PA, USA} % Location
	{Expected: Aug. 2019} % Date
	{
		\begin{cvitems} % Description(s) bullet points
			\item {Minors: Engineering Mechanics, Mathematics}
			\item {Thesis: Effects of Print Orientation, Fill Density and Size on 3D Printed Structures}
		\end{cvitems}
	}
\end{cventries}
\vspace{-1em}
%-------------------------------------------------------------------------------
%	SECTION TITLE
%-------------------------------------------------------------------------------
\cvsection{Certifications}

%-------------------------------------------------------------------------------
%	CONTENT
%-------------------------------------------------------------------------------
\begin{cvhonors}
	%---------------------------------------------------------
	\cvhonor
	{Solidworks Essentials |} % Certificate
	{Prism Engineering} % Granted By
	{Pittsburgh, PA, USA} % Location
	{MAY 2016} % Date(s)
	%---------------------------------------------------------
	\cvhonor
	{Siemens TIA Portal Programming 2 |} % Certificate
	{AWC, Inc.} % Granted By
	{Houston, TX, USA} % Location
	{APR 2016} % Date(s)
	%---------------------------------------------------------
	\cvhonor
	{NFPA 70E |} % Certificate
	{Steel City Safety | Expires: DEC 2017} % Granted By
	{Pittsburgh, PA, USA} % Location
	{DEC 2015} % Date(s)
\end{cvhonors}
\vspace{-1em}
%-------------------------------------------------------------------------------
%	SECTION TITLE
%-------------------------------------------------------------------------------
\cvsection{Professional Memberships}

\vspace{-.5em}
%-------------------------------------------------------------------------------
%	CONTENT
%-------------------------------------------------------------------------------
\begin{cvhonors}

%---------------------------------------------------------
	\cvhonor
	{} % Organiztion
	{International Game Developers Association,  IEEE Computer Society,  ACM,  ASME} % etc
	{} % Joined
	{2017} % Acronym	

%---------------------------------------------------------
\end{cvhonors}
%\newpage
%%-------------------------------------------------------------------------------
%	SECTION TITLE
%-------------------------------------------------------------------------------
\cvsection{Education Comparison}

%-------------------------------------------------------------------------------
%	CONTENT
%-------------------------------------------------------------------------------
\begin{cventries}
	
	%---------------------------------------------------------
	\cventry
	{DEGREE} % Degree
	{INSTITUTION} % Institution
	{LOCATION} % Location
	{DATE} % Date
	{
		\begin{cvitems} % Description(s) bullet points
			\item {ITEM}
		\end{cvitems}
	}
	
	%---------------------------------------------------------
	\cventry
	{DATE} % Date
	{DEGREE} % Degree
	{INSTITUTION} % Institution
	{LOCATION} % Location
	{
		\begin{cvitems} % Description(s) bullet points
			\item {ITEM}
		\end{cvitems}
	}
	
\end{cventries}
%%-------------------------------------------------------------------------------
%	SECTION TITLE
%-------------------------------------------------------------------------------
\cvsection{Experience Comparison}

%-------------------------------------------------------------------------------
%	CONTENT
%-------------------------------------------------------------------------------
\begin{cventries}
	%---------------------------------------------------------
	\cventry
	{TITLE} % Job title
	{ORGANIZATION} % Organization
	{LOCATION} % Location
	{DATES} % Date(s)
	{
		\begin{cvitems} % Description(s) of tasks/responsibilities
			\item {\textbf{HEADER} | {\color{awesome}SKILLS}
				\begin{itemize}[noitemsep,wide=0pt, leftmargin=\dimexpr\labelwidth + 2\labelsep\relax]
					\item[\textbullet]{}
				\end{itemize}}
		\end{cvitems}
	}
	%---------------------------------------------------------
	\cventry
	{DATES} % Date(s)
	{TITLE} % Job title
	{ORGANIZATION} % Organization
	{LOCATION} % Location
	{
		\begin{cvitems} % Description(s) of tasks/responsibilities
			\item {\textbf{HEADER} | {\color{awesome}SKILLS}
				\begin{itemize}[noitemsep,wide=0pt, leftmargin=\dimexpr\labelwidth + 2\labelsep\relax]
					\item[\textbullet]{}
				\end{itemize}}
		\end{cvitems}
	}
\end{cventries}

%-------------------------------------------------------------------------------
\end{document}